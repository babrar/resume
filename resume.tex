% Changes to be made 
% Include Additional Skills in skills section
% Include other skills in First section
% Include agile for work 
% Make the projects bullet pointed (add more lines)
% Make relevant courses contiuous line
%%%%%%%%%%%%%%%%%%%%%%%%%%%%%%%%%%%%%%%
% This is a modified ONE COLUMN version of
% the following template:
% 
% Deedy - One Page Two Column Resume
% LaTeX Template
% Version 1.1 (30/4/2014)
%
% Original author:
% Debarghya Das (http://debarghyadas.com)
%
% Original repository:
% https://github.com/deedydas/Deedy-Resume
%
% IMPORTANT: THIS TEMPLATE NEEDS TO BE COMPILED WITH XeLaTeX
%
% This template uses several fonts not included with Windows/Linux by
% default. If you get compilation errors saying a font is missing, find the line
% on which the font is used and either change it to a font included with your
% operating system or comment the line out to use the default font.
% 
%%%%%%%%%%%%%%%%%%%%%%%%%%%%%%%%%%%%%%
% 
% TODO:
% 1. Integrate biber/bibtex for article citation under publications.
% 2. Figure out a smoother way for the document to flow onto the next page.
% 3. Add styling information for a "Projects/Hacks" section.
% 4. Add location/address information
% 5. Merge OpenFont and MacFonts as a single sty with options.
% 
%%%%%%%%%%%%%%%%%%%%%%%%%%%%%%%%%%%%%%
%
% CHANGELOG:
% v1.1:
% 1. Fixed several compilation bugs with \renewcommand
% 2. Got Open-source fonts (Windows/Linux support)
% 3. Added Last Updated
% 4. Move Title styling into .sty
% 5. Commented .sty file.
%
%%%%%%%%%%%%%%%%%%%%%%%%%%%%%%%%%%%%%%%
%
% Known Issues:
% 1. Overflows onto second page if any column's contents are more than the
% vertical limit
% 2. Hacky space on the first bullet point on the second column.
%
%%%%%%%%%%%%%%%%%%%%%%%%%%%%%%%%%%%%%%

\documentclass[]{deedy-resume-openfont}


\begin{document}


%%%%%%%%%%%%%%%%%%%%%%%%%%%%%%%%%%%%%%
%
%     TITLE NAME
%
%%%%%%%%%%%%%%%%%%%%%%%%%%%%%%%%%%%%%%


\namesection{Banin}{Abrar}{ \urlstyle{same}\url{babrar.github.io} \\
\href{mailto:babrar@edu.uwaterloo.ca}{babrar@edu.uwaterloo.ca} |  519-729-6017 
}
\section{Summary}
\runsubsection{Strengths}
\descript{}
%\location{Expected June 2014 – Sep 2014 | Mountain View, CA}
%\vspace{\topsep} % Hacky fix for awkward extra vertical space
\sectionsep
\begin{tightemize}
\item Proficient in C, C++, JavaScript, Python, Perl, LaTeX, HTML, CSS, Bash.
\item Experienced with Git, SVN, RHEL, SIMD and basic familiarity with AWS (EC2 and S3). 
%\item Exposure to Digital Design through Verilog and VHDL.
\end{tightemize}
\sectionsep

\runsubsection{Interests}
\sectionsep
\begin{tightemize}
\item Backend Development, Algorithm Optimization.
\end{tightemize}
\sectionsep
\section{Experience}
\runsubsection{Integrated Device Technology (IDT)}
\descript{| Algorithm Engineer \\ \color{gray}{Jan 2018 – Apr 2018 | Waterloo, ON} }
%\location{Jan 2018 – Apr 2018 | Waterloo, ON}
\begin{tightemize}
\item Improved H.265 encoder's load distribution across CPU cores by implementing Intrinsics on C-code.
\item Increased pixel metadata collection speed by a factor of 4, over GCC’s default auto-vectorization.
\item Analyzed potential benefits of using 512-bit instruction sets in IDT’s x86 mainframe.
\item Optimized pipeline designs in FPGA using Verilog. Reduced total negative slack by 30 percent.
\item Wrote automation scripts to improve SSH compatibility in the internal workflow of the company.
\item Completed tasks and projects using Agile Development, through SCM tools, i.e. JIRA.
\end{tightemize}
\sectionsep
%%%%%%%%%%%%%%%%%%%%%%%%%%%%%%%%%%%%%%
%     Projects
%%%%%%%%%%%%%%%%%%%%%%%%%%%%%%%%%%%%%%

\section{Projects}

\runsubsection{F-Search}
\descript{| Python \\ \color{gray}{Current Project} }
%\location{Upcoming Project}
\begin{tightemize}
\item F-Search is a speed-efficient extended fuzzy search algorithm.
\item Generating Trie to map words with frequency of use. Inspired by \href{http://norvig.com/mayzner.html}{\underline{Peter Norvig's Research Post}}.
\item Improving suggestion accuracy through the use of sequence transposition algorithm.
%\item Use of Damerau–Levenshtein distance over general LD algorithm to allow for consideration of sequence transpositions.
\end{tightemize}
\sectionsep

\runsubsection{Smart-Cane}
\descript{| C++ \\ \color{gray}{Oct 2017}}
%\location{Oct 2017}
\begin{tightemize}
\item Smart-Cane is a walking stick designed for the visually-impaired. 
\item Simulated a proximity sensor by utilizing data collected from a short-range ultrasonic sensor. 
\item Automated data collection and implemented an alarm system to warn the user of the cane \\ in case of close proximity to obstacles.
\item Generated MIPS compatible instructions through cross-compilation of C++.
\end{tightemize}
\sectionsep

%%%%%%%%%%%%%%%%%%%%%%%%%%%%%%%%%%%%%%
%     education
%%%%%%%%%%%%%%%%%%%%%%%%%%%%%%%%%%%%%%
\section{Education}
\runsubsection{University of Waterloo}
\descript{| Candidate for B.A.Sc in Honours Computer Engineering \\ \color{gray}{Expected Apr 2022 | Waterloo, ON}}
%\location{Expected Apr 2022 | Waterloo, ON}
\begin{tightemize}
\item President’s Scholarship of Distinction (2017)
\end{tightemize}
\sectionsep

\descript{\color{gray}{Relevant Coursework}}
\sectionsep
\textbullet{} Fundamentals of Programming (ECE 150) \textbullet{}   Discrete Mathematics and Logic 1 (ECE 108) \\
\textbullet{} Digital Circuits and Systems (ECE 124) \textbullet{} edX Computer Science (CS50) \\ \textbullet{} Programming in C++ (New Horizons)
\sectionsep


\end{document}  \documentclass[]{article}